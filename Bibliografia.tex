\documentclass[TFG.tex]{subfiles}
\cleardoublepage %para la versión impresa esto dependiendo del número de páginas puede ser mejor quitarlo pero para el pdf es mejor dejarlo
\phantomsection
\addcontentsline{toc}{chapter}{Bibliografía}

\begin{document}
%\renewcommand\chaptername{\Huge Tema}
%
%\titleformat{\chapter}[display]
%    {\normalfont\huge\bfseries}{\chaptertitlename\ \thechapter}{10pt}{\Huge}
%\titlespacing*{\chapter}{0pt}{-1cm}{10pt}

\begin{thebibliography}{}
\bibitem{Alexander} J. W. Alexander. \emph{On the deformation of an n-cell}. Proc. of the Nat.
Acad. of Sci. of the USA., 9 (12):406–407, 1923. \url{https://www.ncbi.nlm.nih.gov/pmc/articles/PMC1085470/pdf/pnas01885-0010.pdf}
\bibitem{ArtinA} E. Artin. \emph{Theorie der Zöpfe}. Abh. Math. Sem. Hamburgischen Univ.,
4:47–72, 1925.
\bibitem{Artin} E. Artin. \emph{The theory of braids}. Annals of Math., 48:101–126, 1947. \url{http://www.maths.ed.ac.uk/~aar/papers/artinbraids.pdf}

\bibitem{invariants} M.A. Berger. \emph{Topological invariants in braid theory}.  Letters in Mathematical Physics (2001) 55: 181. \url{https://doi.org/10.1023/A:1010979823190}
\bibitem{Bigelow} S.J. Bigelow, \emph{Braid groups are linear}, J. Amer. Math. Soc. 14 (2001), no. 2,
471–486. \url{https://arxiv.org/pdf/math/0005038.pdf}
\bibitem{Bil} S. Bigelow. \emph{The Burau representation is not faithful for $n = 5$}. Geom. Topol. 3 (1999), no. 1, 397--404. doi:10.2140/gt.1999.3.397. 
\url{https://arxiv.org/pdf/math/9904100.pdf}
\bibitem{Birman} J. S. Birman. \emph{Braids, links and mapping class groups}. Annals of Mathematics
Studies, No. 82. Princeton University Press, Princeton, N.J.,
1974.
\bibitem{Burau} Burau, Werner (1936). \emph{Über Zopfgruppen und gleichsinnig verdrillte Verkettungen}. Abh. Math. Sem. Univ. Hamburg. 11: 179–186. doi:10.1007/bf02940722

\bibitem{Clifford}  A. H. Clifford and G.B. Preston. \emph{The algebraic theory of semigroups}. AMS Math. Surveys,
vol. 7, 1967. MR0218472 (36:1558)

\bibitem{Dehn12} M. Dehn. \emph{Transformation der Kurven auf zweiseitigen Flächen}. Mathematische Annalen (1912), 72 (3): 413–421
\bibitem{Dehn11} M. Dehn. \emph{Über unendliche diskontinuierliche Gruppen}. Mathematische Annalen (1911), 71 (1): 116–144.
\bibitem{Dehornoy} P. Dehornoy. \emph{Groupes de Garside}. Annales scientifiques de l'École Normale Supérieure, Série 4 : Tome 35 (2002) no. 2 , p. 267-306 \url{https://arxiv.org/pdf/math/0111157.pdf}
\bibitem{Dynnikov} P. Dehornoy, I. Dynnikov, D. Rolfsen, B. Wiest. \emph{Ordering Braids}. Mathematical Surveys and Monographs
Volume: 148; 2008; 323 pp;  ISBN: 978-0-8218-4431-1.

\bibitem{EM}E. A. El-Rifai, H. R. Morton. \emph{Algorithms for positive braids}.
Quart. J. Math. Oxford Ser. (2), 45 (180):479–497, 1994.
\bibitem{Thurston} D. B. A. Epstein, J. W. Cannon, D. F. Holt, S. V. F. Levy, M. S.
Paterson, W. P. Thurston. \emph{Word processing in groups}. Jones and
Bartlett Publishers, Boston, MA, 1992.

\bibitem{Fadell}  E. Fadell, L. Neuwirth. \emph{Configuration spaces}. Math. Scand.,
10:111–118, 1962. \url{http://www.mscand.dk/article/download/10517/8538}

\bibitem{Garside} F. A. Garside. \emph{The braid group and other groups}. Quart. J. Math.
Oxford Ser. (2), 20:235–254, 1969. \url{http://www.maths.ed.ac.uk/~aar/papers/garside.pdf}
\bibitem{Meneses} J. González-Meneses. \emph{Basic results on braid groups}. Annales mathématiques, Blaise Pascal Working version – October 5, 2010. \url{https://arxiv.org/abs/1010.0321v1}
\bibitem{polynomial} J. González-Meneses, M. Silvero. \emph{Polynomial braid combing}. December 5, 2017. \url{https://arxiv.org/abs/1712.01552}

\bibitem{Hatcher} A. Hatcher, \emph{Algebraic Topology}. Cambridge University Press, 2002. \url{https://www.math.cornell.edu/~hatcher/AT/AT.pdf}
\bibitem{Hur} A. Hurwitz. \emph{Über Riemannsche Flächen mit gegebenen Verzweigungspunkten}.
Math. Ann., 39 (1):1–60, 1891.

\bibitem{thesis} C. H. Jackson, \emph{Braid group representations}. The Ohio State University
2001. \url{http://go.owu.edu/~chjackso/Papers/thesis.pdf}

\bibitem{Krammer} D. Krammer, \emph{Braid groups are linear}, Ann. of Math. (2) 155 (2002), no. 1,
131–156. \url{https://arxiv.org/pdf/math/0405198.pdf}

\bibitem{Magnus} W. Magnus. \emph{Über Automorphismen von Fundamentalgruppen berandeter
Flächen}. Math. Ann., 109:617–646, 1934.
\bibitem{Markoff} A. Markov. \emph{Foundations of the algebraic theory of tresses}. (russian).
Trav. Inst. Math. Stekloff, 16:53 pp., 1945. \url{http://www.mathnet.ru/links/f1a9f74975a0ed7e11860df5c4a69c58/tm911.pdf}

\bibitem{Novikov}  P. S. Novikov. \emph{On the algorithmic unsolvability of the word problem in group theory}, Proceedings of the Steklov Institute of Mathematics (in Russian), 44: 1–143 (1955). 

\bibitem{Ore} O. Ore. \emph{Linear equations in non-commutative fields}. Ann. of Math.
(2), 32 (3):463–477, 1931.

\bibitem{Luis} L. Paris. \textit{Braid groups and Artin groups}. Institut de Mathématiques de Bourgogne – November 15, 2007. \url{https://arxiv.org/abs/0711.2372v1}
\bibitem{LP}D. D. Long, M. Paton.
\emph{The Burau representation is not faithful for $n ≥ 6$}.
Topology,
Volume 32, Issue 2,
1993,
Pages 439-447,
ISSN 0040-9383. \url{http://web.math.ucsb.edu/~long/pubpdf/Burau_n6.pdf}

\bibitem{problemas} V. Shpilrain. \emph{Search and witness problem in group theory}. October 3, 2010. \url{https://arxiv.org/abs/1010.0382}

\bibitem{nundam} V. Turaev. \emph{Faithful linear representations of the braid groups}. Séminaire N. Bourbaki, 1999-2000, exp. nº 878, p.389-409. \url{http://www.numdam.org/article/SB_1999-2000__42__389_0.pdf}

\bibitem{Zariski} O. Zariski. \emph{On the Poincaré group of rational plane curves}. Amer. J.
of Math., 58 (3):607–619, 1936.



















\end{thebibliography}

\end{document}