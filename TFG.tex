\documentclass[twoside,12pt]{report}
\usepackage{titlesec}
\usepackage{multirow}
\usepackage{subfig}
%\usepackage{wasysym}
\usepackage{tikz-cd}
\usetikzlibrary{arrows}
\usetikzlibrary{cd}
\usepackage{rotating}
\usetikzlibrary{automata,positioning}
\newcommand{\mychapter}[2]{
    \setcounter{chapter}{#1}
    \setcounter{section}{0}
    \chapter*{#2}
    \addcontentsline{toc}{chapter}{#2}
}

\usepackage{estilo-apuntes}
\usepackage{braids}
\usepackage{pgfplots}
\pgfplotsset{compat=1.13}
\usetikzlibrary{decorations.shapes}
\pgfkeyssetvalue{/tikz/braid height}{1cm} %no parece hacer nada
\pgfkeyssetvalue{/tikz/braid width}{1cm}
\pgfkeyssetvalue{/tikz/braid start}{(0,0)}
\pgfkeyssetvalue{/tikz/braid colour}{black}
\begin{document}
\begin{titlepage}
	\centering
		\includegraphics[width=8cm]{Imagenes/sello.jpeg}
		
	{\Large\bfseries UNIVERSIDAD DE SEVILLA\par}
	{\Large\bfseries FACULTAD DE MATEMÁTICAS\par}
	\vspace{0.5cm}
	{\large\bfseries Trabajo de Fin de Grado \par}
	\vspace{1cm}
	{\Huge\bfseries El problema de la palabra en los grupos de trenzas\par}
	\vspace{1cm}
	{\large Por:\par}
	{\large Javier Aguilar Martín\par}
	\vspace{0.5cm}
	{\large Dirigido por:\par}
	{\large Juan González-Meneses López y Ramón Jesús Flores Díaz\par}
	\vspace{1cm}
	{\large Grado en Matemáticas\par}
	\vspace{0.5cm}
	{\large \today\par}
	\vspace{1cm}





	
\end{titlepage}



\tableofcontents

\subfile{Abstract}
\subfile{Resumen}
\subfile{Definiciones}
\subfile{AutomorfismosDelGrupoLibre}
\subfile{PeinadoDeTrenzas}
\subfile{FormasNormales}
\subfile{RepresentacionesLineales}
\subfile{Bibliografia}

%¿AÑADO ALGUNA COSA MÁS DE ESTILO?











\end{document}
