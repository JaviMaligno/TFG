\documentclass{beamer}
\usepackage[utf8]{inputenc}
\usetheme{Copenhagen}
\usepackage[spanish]{babel}
\usepackage{multirow}
%\usepackage{estilo-apuntes}
\usepackage{braids}
\usepackage[]{graphicx}
\usepackage{rotating}
\usepackage{pgf,tikz}
\usepackage{pgfplots}
\usepackage{tikz-cd}
\usetikzlibrary{arrows}
\usetikzlibrary{cd}
\usetikzlibrary{babel}
\pgfplotsset{compat=1.13}
\usetikzlibrary{decorations.shapes}
\pgfkeyssetvalue{/tikz/braid height}{1cm} %no parece hacer nada
\pgfkeyssetvalue{/tikz/braid width}{1cm}
\pgfkeyssetvalue{/tikz/braid start}{(0,0)}
\pgfkeyssetvalue{/tikz/braid colour}{black}

\theoremstyle{definition}

\newtheorem{teorema}{Teorema}
\newtheorem{defi}{Definición}
\newtheorem{prop}[teorema]{Proposición}

\newcommand{\Z}{\mathbb{Z}}
\newcommand{\C}{\mathbb{C}}
\newcommand{\D}{\mathbb{D}}
\providecommand{\gene}[1]{\langle{#1}\rangle}


\addtobeamertemplate{navigation symbols}{}{%
    \usebeamerfont{footline}%
    \usebeamercolor[fg]{footline}%
    \hspace{1em}%
    \insertframenumber/\inserttotalframenumber
}
\setbeamercolor{footline}{fg=black}
\setbeamerfont{footline}{series=\bfseries}

%-----------------------------------------------------------

\title{El problema de la palabra en los grupos de trenzas}
\author{Javier Aguilar Martín}
\institute{Universidad de Sevilla}
\date{}
 
\begin{document}
\frame{\titlepage}
%\begin{frame}
%
%
%\title[About Beamer] %optional
%{About the Beamer class in presentation making}
% 
%\subtitle{A short story}
% 
%\author[Arthur, Doe] % (optional, for multiple authors)
%{A.~B.~Arthur\inst{1} \and J.~Doe\inst{2}}
% 
%\institute[VFU] % (optional)
%{
%  \inst{1}%
%  Faculty of Physics\\
%  Very Famous University
%  \and
%  \inst{2}%
%  Faculty of Chemistry\\
%  Very Famous University
%}
% 
%\date[VLC 2013] % (optional)
%{Very Large Conference, April 2013}


%\end{frame}
\setbeamercovered{highly dynamic}

\newcounter{saveenumi}
\newcommand{\seti}{\setcounter{saveenumi}{\value{enumi}}}
\newcommand{\conti}{\setcounter{enumi}{\value{saveenumi}}}

\resetcounteronoverlays{saveenumi}
%\AtBeginSection[]{
%\begin{frame}
%\frametitle{Tabla de contenidos}
%\tableofcontents
%\end{frame}
%}
\begin{frame}
\begin{enumerate}
%\item Saludo, agradecimientos, decir que tenía tantas ganas de presentarlo que he venido desde Madrid
\item Presentarlo, decir que explicaré brevemente el problema de la palabra, daré varias visiones de los grupos de trenzas y varios algoritmos (muy por encima) para resolver el problema en estos grupos
\item Problema de la palabra
\seti
\end{enumerate}
\end{frame}
\begin{frame}
\begin{enumerate}
\conti
\item Trenzas: idea intuitiva de trenza (incluyendo trenzas puras) y concatenación. Esto da un grupo de clases de homotopía relativa.
\item Espacios de configuración: n puntos distintos moviéndose en un espacio sin chocarse, si no importa el orden de los puntos hacemos el cociente. Las trenzas (puras) serían grupo fundamental de estos en C o R2 da lo mismo. 
\item Mapping class group: automorfismos del disco deforman lazos y esto hacen el efecto de cruce de puntos, que se corresponde con los cruces de cuerdas (Poner imagen de automorfismos del grupo libre) 
\item Presentación del grupo: generadores son cruces, también tenemos presentación para las trenzas puras con generadores de Birman (dibujo)
\seti
\end{enumerate}
\end{frame}
\begin{frame}
\begin{enumerate}
\conti
\item Automorfismos del grupo libre :Con la imagen y la expresión de la representación, aquí usamos los mapping class groups, y asociamos un automorfismo del grupo libre a cada generador. Esta representación es fiel y en el grupo libre se puede resolver el problema de la palabra. 
\item Peinado de trenzas: Usamos una propiedad de los grupos de trenzas puras, que es que son isomorfos a un producto semidirecto de grupos libres y entonces basta resolver en cada factor el problema. Para expresarlo se usan los generadores de Birman. Para una trenza general primero se comprueba si es pura. (Producto semidirecto: extensión escindida de grupos)
\seti

\end{enumerate}
\end{frame}
\begin{frame}
\begin{enumerate}
\conti
\item Forma normal: Monoide de trenzas positivas: trenzas que tienen exponentes positivas. Aquí existe un orden parcial que admite mínimo común múltiplo y máximo común divisor. Existe un elemento especial que es el mínimo común múltiplo de los generadores. A partir de esto llegamos a una forma normal (la forma normal de Garside) y ya dos palabras representan el mismo elemento si y solo si tienen la misma forma normal. 
(Si tengo que poner en la misma diapositiva el monoide y el Delta hacer para que no salgan a la vez, posiblemente con itemize haciendo que no salga el bullet) BN+ misma presentación con un +
\item Representaciones lineales: se intenta representar las trenzas en un grupo lineal. Enumerar las que menciono en el trabajo diciendo que la de Burau no se sabe si es fiel para n=4 pero las otras sí. 
%\item Finalización: y con esto hemos visto los principales algoritmos. Agradecimiento y preguntas. 
\end{enumerate}
\end{frame}

\begin{frame}
BOTELLA DE AGUA

PUNTERO PARA PASAR DIAPOSITIVAS
\end{frame}
\begin{frame}
LOS AIJ SOLO DIBUJO, QUITAR TODO LO DE NOTACIONES PARA EXPLICARLO HABLANDO. 
EN LINEALES NO HACERLO ESPECÍFICO.
INTENTAR NO MÁS DE 2 POR CADA PARTE.


\end{frame}
\section{El problema de la palabra}

\begin{frame}
\frametitle{El problema de la palabra}
Sea $G=\gene{S\mid R}$. Dados $A,B\in G$ expresados como producto de los generadores de $G$ y sus inversos
\[
\mbox{¿}A=B?\Leftrightarrow \mbox{¿}AB^{-1}=1?
\]
\end{frame}

\section{Definiciones de los grupos de trenzas}

\subsection{Trenzas geométricas}

\begin{frame}
\frametitle{Trenzas geométricas}
\begin{figure}[h!]
\includegraphics[scale=0.5]{Imagenes/hilos}
\caption{Una trenza geométrica pura y una trenza geométrica no pura.}
\end{figure}
\end{frame}

\begin{frame}[fragile]
\frametitle{Representación plana}
\begin{figure}[h!]
\resizebox{11cm}{2.5cm}{
\begin{tikzpicture}
\fill[black] ( 6.7 , -0.5 ) circle ( 1 pt ) ;
\fill[black] ( 7 , -0.5 ) circle ( 1 pt ) ;
\fill[black] ( 7.3 , -0.5 ) circle ( 1 pt ) ;

\fill[black] ( 1.7 , -0.5 ) circle ( 1 pt ) ;
\fill[black] ( 2 , -0.5 ) circle ( 1 pt ) ;
\fill[black] ( 2.3 , -0.5 ) circle ( 1 pt ) ;
\braid[ number of strands=8, %lower bound
 border height=2pt,style strands={8}{line width=2pt}, style strands={7}{draw=none},style strands={6}{line width=2pt},style strands={5}{line width=2pt},style strands={4}{line width=2pt},style strands={3}{line width=2pt}, style strands={2}{draw=none},style strands={1}{line width=2pt}] (braid) at (1,0)%optional, it's a name
          a_4^{-1};
          \node[ at=(braid-4-s),  label=north :  $i$ ] {} ;
\node[ at=(braid-5-s),  label=north : $i+1$ ] {} ;
\draw (4.5,-2) node[anchor=south] {$\sigma_i$};
\end{tikzpicture}
\qquad
\begin{tikzpicture}
\fill[black] ( 6.7 , -0.5 ) circle ( 1 pt ) ;
\fill[black] ( 7 , -0.5 ) circle ( 1 pt ) ;
\fill[black] ( 7.3 , -0.5 ) circle ( 1 pt ) ;

\fill[black] ( 1.7 , -0.5 ) circle ( 1 pt ) ;
\fill[black] ( 2 , -0.5 ) circle ( 1 pt ) ;
\fill[black] ( 2.3 , -0.5 ) circle ( 1 pt ) ;
\braid[ number of strands=8, %lower bound
 border height=2pt,style strands={8}{line width=2pt}, style strands={7}{draw=none},style strands={6}{line width=2pt},style strands={5}{line width=2pt},style strands={4}{line width=2pt},style strands={3}{line width=2pt}, style strands={2}{draw=none},style strands={1}{line width=2pt}] (braid) at (1,0)%optional, it's a name
          a_4;
          \node[ at=(braid-4-s),  label=north :  $i$ ] {} ;
\node[ at=(braid-5-s),  label=north : $i+1$ ] {} ;
\draw (4.5,-2) node[anchor=south] {$\sigma_i^{-1}$};
\end{tikzpicture}
}
\caption{Cruce positivo y cruce negativo.}
\end{figure}
\end{frame}

\begin{frame}
\begin{figure}[h!]
\includegraphics[scale=0.6]{Imagenes/Diapplana}
\caption{Representación plana de una trenza.}
\end{figure}
\end{frame}

\begin{frame}
\frametitle{Concatenación}
\begin{figure}[h!]
\includegraphics[scale=0.5]{Imagenes/Diapconca}
\caption{Concatenación de trenzas.}
\end{figure}
Esta operación induce un producto entre las clases de homotopía.
\end{frame}

\begin{frame}
Denotemos $B_n$ al conjunto de clases de homotopía relativa a los extremos de trenzas de $n$ cuerdas y $PB_n$ al conjunto de clases de homotopía relativa a los extremos de trenzas puras de $n$ cuerdas.

\begin{prop}
El conjunto $B_n$ dotado de la operación inducida por la concatenación tiene estructura de grupo, al que se le llama \textbf{grupo de trenzas} de $n$ cuerdas. El resultado también es cierto para $PB_n$, cuyo nombre es \textbf{grupo de trenzas puras} de $n$ cuerdas. 
\end{prop}
\end{frame}


\subsection{Espacios de configuración}

\begin{frame}
\frametitle{Espacios de configuración}
\begin{defi}
Dado un espacio topológico $X$, el $n$-ésimo \textbf{espacio de configuración} de $X$ se define como el conjunto
$$M_n(X)=\{(x_1,\dots,x_n)\in X^n\mid x_i\neq x_j\ \forall i\neq j\}$$
dotado de la topología relativa de $X^n$.
\end{defi}
\begin{defi}
Se define el $n$-ésimo \textbf{espacio de configuración no ordenado} de $X$ como el espacio de órbitas
$$N_n(X)=M_n(X)/\Sigma_n$$
\end{defi}
\end{frame}

\begin{frame}
NO SÉ SI PONER TAMBIÉN AQUÍ LAS REFERENCIAS 
\begin{teorema}
El grupo de trenzas puras de $n$ cuerdas es
$$PB_n=\pi_1(M_n(\C))$$
y el grupo de trenzas de $n$ cuerdas es
$$B_n=\pi_1(N_n(\C))$$
\end{teorema}
\end{frame}

\subsection{Mapping class groups}

\begin{frame}
\frametitle{Mapping class groups}
\begin{defi}
Una \textbf{isotopía} entre dos espacios topológicos $X$ e $Y$ es una familia continua de homeomorfismos $h_t:X\to Y$ con $0\leq t\leq 1$. Dos homeomorfismos $f,g:X\to Y$ son \textbf{isotópicos} si existe una isotopía $h_t:X\to Y$ con $h_0=f$ y $h_1=g$. 
\end{defi}
Sea $Homeo^+(\D_n)$ el conjunto de homeomorfismos que preservan la orientación de $\D_n$ en sí mismo, fijando los puntos del borde. 
\end{frame}


\begin{frame}
Consideraremos que dos automorfismos son iguales si pueden ser transformados el uno en el otro mediante una isotopía de $\D_n$ en sí mismo que fije el borde y los agujeros.
\begin{defi} Si denotamos $Homeo^+_0(\D_n)$ a la clase de equivalencia de $Id_{\D_n}$ en $Homeo^+(\D_n)$, definimos entonces el \textbf{mapping class group} (grupo de clases de aplicaciones) de $\D_n$ como:
$$\mathcal{M}(\D_n)=Homeo^+(\D_n)/Homeo^+_0(\D_n)$$
\end{defi}
\begin{teorema}
$$B_n\cong \mathcal{M}(\D_n)$$
\end{teorema}
\end{frame}

\subsection{Presentación del grupo de trenzas}



\begin{frame}
\frametitle{Presentación del grupo de trenzas}
\begin{teorema}
\[
B_n=\left\langle\begin{array}{c| c c}
\multirow{2}{*}{$\sigma_1,\dots,\sigma_{n-1}$} & \sigma_i\sigma_j=\sigma_j\sigma_i, & |i-j|>1\\
& \sigma_i\sigma_j\sigma_i=\sigma_j\sigma_i\sigma_j, & |i-j|=1
\end{array}\right\rangle
\]
\end{teorema}
\end{frame}

\begin{frame}
\frametitle{Presentación del grupo de trenzas puras}
Se definen los generadores 
\[
A_{ij}=\sigma_{j-1}\dots\sigma_{i+1}\sigma_i^2\sigma_{i+1}^{-1}\dots\sigma_{j-1}^{-1}\ (1\leq i<j\leq n)
\]
y las relaciones
\begin{align*}
A_{ij}^{-1}A_{rs}A_{ij}&=A_{rs}\text{ si } (i<j<r<s)\text{ o bien } (r+1<i<j<s),\\
A_{ij}^{-1}A_{js}A_{ij}&=A_{is}A_{js}A_{is}^{-1} \text{ si } (i<j<s),\\
A_{ij}^{-1}A_{is}A_{ij}&=A_{is}A_{js}A_{is}A_{js}^{-1}A_{is}^{-1}\text{ si } (i<j<s),\\
A_{ij}^{-1}A_{rs}A_{ij}&=A_{is}A_{js}A_{is}^{-1}A_{js}^{-1}A_{rs}A_{js}A_{is}A_{js}^{-1}A_{is}^{-1}\text{ si } (i+1<r<j<s).
\end{align*}
\end{frame}

\section{Automorfismos del grupo libre}
\begin{frame}
\frametitle{Automorfismos del grupo libre}

\begin{figure}[h!]
\includegraphics[scale=0.7]{Imagenes/Disco.png}
\caption{Los lazos $x_1,\dots,x_n$ son generadores de $\pi_1(\D_n)\cong F_n$.}
\end{figure}
\end{frame}

\begin{frame}
\begin{align*}
\rho: & B_n \to Aut(F_n)\\
      & \beta\ \ \mapsto\ \ \rho_\beta
\end{align*}

$$\rho_{\sigma_i}(x_i)=x_{i+1},\quad \rho_{\sigma_i}(x_{i+1})=x_{i+1}^{-1}x_ix_{i+1},\quad \rho_{\sigma_i}(x_j)=x_j\ (j\neq i,i+1)$$

$$\rho^{-1}_{\sigma_i}(x_i)=x_ix_{i+1}x_i^{-1},\quad \rho^{-1}_{\sigma_i}(x_{i+1})=x_i,\quad \rho^{-1}_{\sigma_i}(x_j)=x_j\ (j\neq i,i+1)$$
\end{frame}

\begin{frame}
\begin{figure}[h!]
\includegraphics[scale=0.5]{Imagenes/auto.png}
\caption{Acción de $\sigma_i$ sobre los generadores $x_i$ y $x_{i+1}$.}
\end{figure}
\end{frame}

\begin{frame}
\frametitle{Solución al problema de la palabra}
\begin{teorema}
La representación anterior es fiel, es decir, dos trenzas están representadas por el mismo automorfismo si y solo si son la misma.
\end{teorema}

\begin{itemize}
\item Dadas $\beta_1,\beta_2\in B_n$, 
$$\beta_1=\beta_2\Leftrightarrow \rho_{\beta_1}(x_i)=\rho_{\beta_2}(x_i)\ \forall x_i$$
\end{itemize}
\end{frame}

\section{Peinado de trenzas}
\begin{frame}[fragile]
\frametitle{Peinado de trenzas}
\[
\begin{tikzcd}
1\arrow[r]& PB_n\arrow[r, "i"] & B_n\arrow[r,"\eta"]& \Sigma_n\arrow[r] & 1
\end{tikzcd}
\]

\[
\begin{tikzcd}
1\arrow[r]& F_n\arrow[r, "\iota"] & PB_{n+1}\arrow[r,"\rho", shift left]& \arrow[l,"s",shift left]PB_n\arrow[r] & 1
\end{tikzcd}
\]
\end{frame}

\begin{frame}
Inductivamente, combinando $PB_{n+1}\cong F_n\rtimes PB_n$ y $PB_2\cong F_1$ obtenemos:

\begin{block}{Resultado}
$$PB_{n+1}\cong F_n\rtimes (F_{n-1} \rtimes\dots\rtimes (F_3\rtimes (F_2\rtimes F_1))\cdots))$$
\end{block}

\begin{enumerate}
\item<1-> Una trenza $\beta$ es pura si y solo $\eta(\beta)=1$.
\item<2-> Un elemento de un producto semidirecto es trivial si y solo si lo es cada uno de sus coordenadas.
\end{enumerate}
\end{frame}

\begin{frame}
\frametitle{Solución al problema de la palabra}
\begin{enumerate}
\item<1-> Comprobar si $\beta$ es pura calculando la permutación inducida.
\item<2-> Si no es pura, no es trivial. Si es pura, expresarla como elemento del producto semidirecto usando los generadores $A_{ij}$. A esto se le llama ``peinar la trenza''.
\end{enumerate}

\end{frame}

\begin{frame}


\begin{figure}
\begin{turn}{1.5}
\includegraphics[scale=0.4]{Imagenes/peinado}
\end{turn}
\caption{Trenza peinada.}
\end{figure}

\end{frame}



\section{Formas normales}
\begin{frame}
\frametitle{Formas normales}
\begin{defi}
Definimos el monoide $B_n^+$ formado por las trenzas que solo tienen exponentes positivos en los generadores, llamado \textbf{monoide de las trenzas positivas} y sus palabras son llamadas \textbf{trenzas positivas}.
\end{defi}
\begin{defi}
Definimos en $B_n^+$ el orden parcial $\preccurlyeq$ tal que dadas $a,b\in B_n^+$, $a\preccurlyeq b$ si $ac=b$ para alguna $c\in B_n^+$. Decimos en ese caso que $a$ es un \textbf{prefijo} de $b$. %Escribimos $a\prec b$ si $c$ no es trivial. Si además $a\neq 1$, decimos que $a$ es un \emph{prefijo propio} de $b$. 
\end{defi}
\end{frame}

\begin{frame}

\begin{defi}
La \textbf{trenza fundamental} de $n$ cuerdas es la trenza
$$\Delta=\sigma_1(\sigma_2\sigma_1)\cdots(\sigma_{n-1}\cdots\sigma_1)$$
\end{defi} 
\begin{figure}[h!]
\centering
\begin{tikzpicture}
\braid[rotate=90,width=.6cm,height=0.7cm,line width=1.5pt] s_1^{-1} s_2^{-1} s_1^{-1} s_3^{-1} s_2^{-1} s_1^{-1} s_4^{-1} s_3^{-1} s_2^{-1} s_1^{-1};
\end{tikzpicture}
\caption{La trenza fundamental para $n=5$.}
\end{figure}
\end{frame}

\begin{frame}
\frametitle{Forma normal de Garside}

\begin{teorema}
Todo elemento $w\in B_n$ se puede escribir \emph{de forma única} como $w=\Delta^pA$, donde $p\in\Z$, $A\in B_n^+$ y $\Delta\not\preccurlyeq A$. A esta forma de escribir los elementos la denominamos \textbf{forma normal de Garside}.
\end{teorema}

\begin{itemize}
\item Para averiguar si dos trenzas $\beta_1,\beta_2\in B_n$ representan el mismo elemento, basta calcular sus formas normales de Garside y comprobar si coinciden. 
\end{itemize}
\end{frame}



\section{Representaciones lineales}
%\begin{frame}
%\frametitle{Representaciones lineales}
%\end{frame}


\subsection{Representación de Burau}
\begin{frame}
\frametitle{Representación de Burau}

¿PONGO DIRECTAMENTE LAS EXPRESIONES MATRICIALES O HABLO ALGO DE LA CONSTRUCCIÓN?
\end{frame}

\subsection{Representación LKB}
\begin{frame}
\frametitle{Representación LKB}
\end{frame}



\end{document}
