\documentclass[TFG.tex]{subfiles}

\begin{document}



\chapter{Formas normales}\label{capitulo3}

El objetivo de este capítulo será proporcionar una \emph{forma normal} para las trenzas, es decir, una forma ``estándar'' de escribirlas, de modo que para ver si dos palabras representan la misma trenza sea suficiente calcular sus formas normales y comprobar si son iguales. Para llegar hasta esa forma normal estudiaremos la \emph{estructura de Garside} del grupo de trenzas. Los resultados referentes a \emph{Word processing in groups} \cite{Thurston} se pueden encontrar en el capítulo 9 de dicho libro.

\section{Estructura de Garside}

%\url{http://chaoxuprime.com/posts/2011-07-05-garside-normal-form-and-summit-sets.html}


Obsérvese que la presentación \ref{presentacion} solo involucra potencias positivas de los generadores. Por tanto, se puede considerar el monoide $B_n^+$ determinado por esa misma presentación. Los elementos de $B_n^+$ son palabras en $\sigma_1,\dots,\sigma_{n-1}$ (pero no sus inversos), y dos palabras son equivalentes si y solo si una puede obtenerse de la otra reemplazando reiteradamente subpalabras de la forma $\sigma_i\sigma_j$ con $|i-j|>1$ (respectivamente, $\sigma_i\sigma_j\sigma_i$ con $|i-j|=1$) por $\sigma_j\sigma_i$ (respectivamente, $\sigma_j\sigma_i\sigma_j$).

\begin{defi}
El monoide $B_n^+$ se denomina \emph{monoide de las trenzas positivas} y sus palabras son llamadas \emph{trenzas positivas}.
\end{defi}
En el monoide $B_n^+$ hay un orden parcial natural. 
\begin{defi}
Definimos en $B_n^+$ el orden parcial $\preccurlyeq$ tal que dadas $a,b\in B_n^+$, $a\preccurlyeq b$ si $ac=b$ para alguna $c\in B_n^+$. Decimos en ese caso que $a$ es un \emph{prefijo} de $b$. Escribimos $a\prec b$ si $c$ no es trivial. Si además $a\neq 1$, decimos que $a$ es un \emph{prefijo propio} de $b$. 
\end{defi}

Antes de continuar debemos probar que la relación que hemos definido es realmente un orden parcial.

\begin{lemma}
La relación $\preccurlyeq$ es una relación de orden.
\end{lemma}
\begin{dem}
Dada $x\in B_n^+$ se tiene que $x\preccurlyeq x\cdot 1=x$, por lo que se cumple la propiedad reflexiva. Si $x,y\in B_n^+$ con $x \preccurlyeq y $ e $y \preccurlyeq x$, entonces tenemos que $y=xa$ y $x=yb$ para algunos $a,b\in B_n^+$, así que $y=yba$. Como en el monoide de trenzas positivas las relaciones son homogéneas, tenemos necesariamente que $a=b=1$, por lo que $x=y$, cumpliéndose la propiedad antisimétrica. Por último, supongamos que $x\preccurlyeq y\preccurlyeq z$ para ciertas $x,y,z\in B_n^+$. Entonces $y=xa$ y $z=yb$ para $a,b\in B_n^+$. Sustituyendo, $z=xab$, por lo que $x\preccurlyeq z$, lo que prueba la propiedad transitiva. \QED
\end{dem}
Nótese que $\preccurlyeq$ es invariante por multiplicación a izquierda, esto es, $a\preccurlyeq b$ implica $xa\preccurlyeq xb$ para todo $a,b,x\in B_n^+$. %vale el mismo c

Dado tal orden parcial, uno podría preguntarse si existe un único máximo común divisor o mínimo común múltiplo con respecto a $\preccurlyeq$. Esto es, dadas $a,b\in B_n^+$, ¿existe un único $d\in B_n^+$ tal que $d\preccurlyeq a$, $d\preccurlyeq b$ y $d'\preccurlyeq d$ para todo $d'$ prefijo común de $a$ y $b$? ¿Y existe un único $m\in B_n^+$ tal que $a\preccurlyeq m$, $b\preccurlyeq m$ y $m\preccurlyeq m'$ para todo $m'$ que tenga a $a$ y a $b$ como prefijos? En tales casos, escribimos $d=a\land b$ y $m=a\lor b$. Nótese que también tendríamos $xd=xa\land xb$ y $xm=xa\lor xb$ para todo $x\in B_n^+$.

\begin{nota}
Análogamente podríamos definir el orden parcial de \emph{sufijos}, $\succcurlyeq$, invariante por multiplicación a derecha. Nótese que este orden no es equivalente al de prefijos, puesto que $b\succcurlyeq a$ no implica en general $a\preccurlyeq b$ ni recíprocamente. Por ejemplo, $\sigma_1\preccurlyeq \sigma_1\sigma_2$, pero claramente $\sigma_1\sigma_2\not\succcurlyeq \sigma_1$.
\end{nota}

El punto clave en el trabajo de Garside fue demostrar mediante métodos elementales que $\sigma_i$ y $\sigma_j$ tienen mínimo común múltiplo en $B_n^+$. En concreto:
\begin{prop}(\cite[Teorema 1.2]{Garside})
El mínimo común múltiplo de los generadores $\sigma_i$ y $\sigma_j$ viene dado por
$$\sigma_i\lor\sigma_j=\begin{cases}
\sigma_i\sigma_j & |i-j|>1,\\
\sigma_i\sigma_j\sigma_i & |i-j|=1.
\end{cases}$$
\end{prop}
%1.2 GARSIDE, la unicidad se ve fácil por la longitud

Garside prueba al mismo tiempo que $B_n^+$ es cancelativo, es decir, $xay=xby$ implica $a=b$ para todo $a,b,x,y\in B_n^+$.

Como las relaciones de \ref{presentacion} son homogéneas, palabras equivalentes en $B_n^+$ tienen la misma longitud, por lo que la longitud de una trenza positiva se define como la longitud de cualquier palabra que la represente. Aunque Garside no lo menciona explícitamente, un argumento inductivo en esta longitud utilizando la cancelatividad permite probar a partir del resultado anterior que todo par de elementos tiene de $B_n^+$ tiene un único mínimo común múltiplo y un único máximo común divisor, tal como se prueba en \cite{Dehornoy}. 
%relaciones homogéneas= LA SUMA DE LOS EXPONENTES A UN LADO Y A OTRO DE LA ECUACIÓN ES LA MISMA
%lo de dehornoy lo tengo en el mismo papel donde me explica el 1.2 de Garside y lo de que los sigma_i son sufijos de Delta. Lo del mcm es lo de las flechitas y el mcd es los bloques

Garside después estudia el siguiente elemento especial.
\begin{defi}
La \emph{trenza fundamental} de $n$ cuerdas es la trenza
$$\Delta_n=\sigma_1(\sigma_2\sigma_1)\cdots(\sigma_{n-1}\cdots\sigma_1).$$
Cuando $n$ se sobreentiende, escribimos simplemente $\Delta$. 
\end{defi} 

\newpage
\begin{prop}\label{conjuga}
Se verifican: 
\begin{enumerate}
\item $\Delta=\sigma_1\lor\dots\lor\sigma_{n-1}$ \cite[Lema 1]{Garside}.
\item $\sigma_i\Delta=\Delta\sigma_{n-i}$ para todo $i=1,\dots, n-1$ \cite[Lema 4]{Garside}.
\end{enumerate}
\end{prop} 


\begin{figure}[h!]
\centering
\begin{tikzpicture}
\braid[rotate=90,width=.65cm,height=0.7cm,line width=1.5pt] s_1^{-1} s_2^{-1} s_1^{-1} s_3^{-1} s_2^{-1} s_1^{-1} s_4^{-1} s_3^{-1} s_2^{-1} s_1^{-1};
\end{tikzpicture}
\caption{La trenza fundamental $\Delta_5$.}
\end{figure}


A partir de este resultado podemos deducir las siguientes propiedades sobre $\Delta$.

\begin{prop}\label{sufijos} Se cumplen:
\begin{enumerate}
\item $\sigma_1,\dots,\sigma_{n-1}$ son también sufijos de $\Delta$.
\item $\Delta^2$ conmuta con todo elemento de $B_n^+$.
\item Para todo $a\in B_n^+$ se tiene $a\preccurlyeq\Delta^m$ y $\Delta^m\succcurlyeq a$, donde  $m\geq 0$ es la longitud de $a$.
\end{enumerate}
 
\end{prop}
\begin{dem}\
\begin{enumerate}
\item En primer lugar, se tiene para $k>i\geq 1$ que $\sigma_i(\sigma_k\cdots\sigma_1)=(\sigma_k\cdots\sigma_1)\sigma_{i+1}$. En efecto, usando las relaciones del monoide de trenzas positivas
\begin{align*}
\sigma_i(\sigma_k\cdots\sigma_1)&=\sigma_i(\sigma_k\cdots\sigma_{i+2})(\sigma_{i+1}\sigma_i)(\sigma_{i-1}\cdots\sigma_1)\\
&=(\sigma_k\cdots\sigma_{i+2})(\sigma_i\sigma_{i+1}\sigma_i)(\sigma_{i-1}\cdots\sigma_1)\\
&=(\sigma_k\cdots\sigma_{i+2})(\sigma_{i+1}\sigma_i\sigma_{i+1})(\sigma_{i-1}\cdots\sigma_1)\\
&=(\sigma_k\cdots\sigma_{i+2})(\sigma_{i+1}\sigma_i)(\sigma_{i-1}\cdots\sigma_1)\sigma_{i+i}\\
&=(\sigma_k\cdots\sigma_1)\sigma_{i+1}.
\end{align*}
Así pues, para expresar $\sigma_i$ como sufijo de $\Delta$ hacemos los siguiente. Si $i=1$, entonces por la definición de $\Delta$ ya tenemos que es un sufijo. Si $1<i\leq n-1$, partimos de
\[
\Delta=\sigma_1(\sigma_2\sigma_1)\cdots(\sigma_{n-i}\dots\underline{\sigma_1})\cdots(\sigma_{n-1}\dots\sigma_1).
\]
Procedemos a desplazar a la derecha el $\sigma_1$ subrayado tal como hemos hecho anteriormente. En cada paso irá aumentando el índice en una unidad. Por tanto, como hay $n-(n-i-1)=i-1$ bloques que se dejan atrás, obtenemos $\sigma_{1+i-1}=\sigma_i$, es decir,
\[
\Delta=\sigma_1(\sigma_2\sigma_1)\cdots(\sigma_{n-i}\dots\underline{\sigma_1})(\sigma_{n-i+1}\cdots\sigma_1)\cdots(\sigma_{n-1}\dots\sigma_1)=
\]
\[
\sigma_1(\sigma_2\sigma_1)\cdots(\sigma_{n-i}\cdots\sigma_2)(\sigma_{n-i+1}\cdots\sigma_1\underline{\sigma}_2)\cdots(\sigma_{n-1}\dots\sigma_1)=
\]
\[
\sigma_1(\sigma_2\sigma_1)\cdots(\sigma_{n-i}\cdots\sigma_2)(\sigma_{n-i+1}\cdots\sigma_1)\cdots(\sigma_{n-1}\dots\sigma_1)\underline{\sigma}_i
\]


\item Basta probar que $\Delta^2$ conmuta con $\sigma_i$ para todo $1\leq i\leq n-1$. Como $\sigma_i\Delta=\Delta\sigma_{n-i}$ y $\sigma_{n-i}\Delta=\Delta\sigma_i$ se tiene
\[
\sigma_i\Delta^2=\Delta\sigma_{n-i}\Delta=\Delta^2\sigma_i.
\]


\item Probamos $a\preccurlyeq\Delta^m$ donde $m$ es la longitud de $a$ por inducción en $m$. Evidentemente, $1\preccurlyeq\Delta^0=1$. Para una palabra de longitud 1 también es claro porque $\sigma_i\preccurlyeq\Delta$ para todo $1\leq i\leq n-1$ por definición de mínimo común múltiplo. Supongamos ahora, que para una palabra $a\in B_n^+$ de longitud $m-1$ se tiene el resultado. Entonces, cualquier palabra de longitud $m$ será de la forma $\sigma_j a$ para algún $1\leq j\leq n-1$. Así que, usando la invarianza por multiplicación a izquierda y el caso $m=1$,
\[
a\preccurlyeq\Delta^{m-1}\Rightarrow \sigma_j a\preccurlyeq \sigma_j\Delta^{m-1} =\Delta^{m-1}\sigma_t\preccurlyeq \Delta^{m-1}\Delta=\Delta^{m}
\]
donde $t=j$ o bien $t=n-j$ dependiendo de la paridad de $m$. De forma análoga usando la invarianza por multiplicación a derecha se prueba que $\Delta^m\succcurlyeq a$.
\end{enumerate}
\QED
\end{dem}


% el a<\Delta^m es cuestión de que como \sigma_i<\Delta, luego ya solo hay que ir multiplicando a izquierda por el sigma_j correspondiente -> \sigma_j\sigma_i<\sigma_j\Delta<\Delta^2, etc El otro igual usando la multiplicación a la derecha

Esto tiene importantes implicaciones. Como todo par de elementos de $B_n^+$ tiene un múltiplo común y $B_n^+$ es cancelativo, las condiciones de Ore (\ref{condiciones}) implican que $B_n^+$ se inyecta en su grupo de fracciones, que es precisamente $B_n$. Por lo tanto, $B_n^+$ no es solamente un monoide definido algebraicamente, sino que puede ser considerado como un submonoide de $B_n$ formado por las trenzas que pueden ser escritas solo con potencias positivas de los generadores. 

Las propiedades anteriores implican que el orden parcial $\preccurlyeq$ (respectivamente, $\succcurlyeq$) puede ser extendido a $B_n$ de la siguiente manera: dadas $a,b\in B_n$, $a\preccurlyeq b$ (resp. $a\succcurlyeq b$) si $ac=b$ (resp. $b=ca$) para algún $c\in B_n^+$. Esto da un orden parcial que es invariante por multiplicación a izquierda (resp. a derecha), y el cual admite un único mínimo común múltiplo y un único máximo común divisor. Este hecho podrá ser probado una vez definida la \emph{forma normal de Garside} en la sección a continuación.
% a,b\in B_n, a<b sii a^{-1}b\in B_n^+

\section{Solución al problema de la palabra}
Garside dio una nueva solución al problema de la palabra en los grupos de trenzas de la siguiente manera. Recordemos que para todo $i=1,\dots, n-1$ se tiene que $\Delta\succcurlyeq\sigma_i$ por la Proposición \ref{sufijos} apartado 1, esto es, $\Delta=X_i\sigma_i$ para algún $X_i\in B_n^+$. Dada una trenza escrita como una palabra en $\sigma_1,\dots,\sigma_{n-1}$ y sus inversos, se puede reemplazar cada aparición de $\sigma_i^{-1}$ por $\Delta^{-1}X_i$. Conjugar una trenza positiva por $\Delta$ sigue dando una trenza positiva por la Proposición \ref{conjuga} apartado 2, así que podemos mover todas las apariciones de $\Delta^{-1}$ a la izquierda, de la siguiente forma: si encontramos $\sigma_j\Delta^{-1}$ ($1\leq j\leq n-1)$, entonces por \ref{conjuga} sabemos que $\Delta\sigma_{j}=\sigma_{n-j}\Delta$, si y solo si  $\sigma_j\Delta^{-1}=\Delta^{-1}\sigma_{n-j}$, por lo que podemos sustituir $\sigma_j\Delta^{-1}$ por $\Delta^{-1}\sigma_{n-j}$. Esto muestra que toda trenza puede ser escrita como $\Delta^p A$ para algún $p\in\Z$ y algún $A\in B_n^+$. Además, si $\Delta\preccurlyeq A$, podemos reemplazar $\Delta^p$ por $\Delta^{p+1}$ y $A$ por $\Delta^{-1}A$. Esto reduce la longitud de $A$, así que solo puede hacerse una cantidad finita de veces. Por tanto, toda trenza puede descomponerse \emph{de manera única}, como $\Delta^pA$, donde $p\in\Z$, $A\in B_n^+$ y $\Delta\not\preccurlyeq A$. Efectivamente, si tuviéramos dos expresiones $\Delta^pA=\Delta^q B$ con $p<q$ en las condiciones anteriores, dividiendo por $\Delta^p$ tendríamos que $A=\Delta^{q-p}B$, lo cual contradice el hecho de que $A$ no tenga a $\Delta$ como prefijo. Análogamente para $p>q$, luego $p=q$ y $A=B$.

%\Delta c=A, \Delta^p A = \Delta^{p+1} \Delta^{-1}A=\Delta^{p+1}c
\begin{defi}
En base a lo comentado en el párrafo anterior, definimos la \emph{forma normal de Garside} de una palabra $w\in B_n$ como $w=\Delta^pA$, donde $p\in\Z$, $A\in B_n^+$ y $\Delta\not\preccurlyeq A$. 
\end{defi}

Esta forma normal permite resolver el problema de la palabra, ya que se pueden enumerar todas las palabras positivas que representan la trenza positiva $A$ reiterando las relaciones del monoide de trenzas positivas de todas las formas posibles. Esta fue la solución dada por Garside en \cite{Garside}. Sin embargo, no es muy satisfactoria, ya que da lugar a un algoritmo altamente ineficiente.

%al ser relaciones positivas y tal (la longitud es fija), solo hay una cantidad finita de posibilidades

El-rifai y Morton \cite{EM} lo mejoraron definiendo la \emph{forma normal a la izquierda} de una trenza. Basta tomar la descomposición $\Delta^pA$ y después definir %izquierda porque se van pasando elementos a la izquierda
\begin{align*}
&a_1=A\land\Delta\\
&a_i=(a_{i-1}^{-1}\cdots a_1^{-1}A)\land\Delta,\ \forall\ i>1.
\end{align*}
Nótese que existe un $r\geq 0$ tal que $a_i=1$ para todo $i>r$, ya que la longitud de $a_{i-1}^{-1}\cdots a_1^{-1}A$ es estrictamente decreciente. De esta forma, toda trenza puede ser escrita de manera única como:
%la longitud decrece porque estás cancelando prefijos de A, luego cada vez lo haces más corto
$$\Delta^p a_1\cdots a_r,$$
donde los $a_i$ son los definidos anteriormente, los cuales por definición son un prefijos propios de $\Delta$, es decir, $1\prec a_i\prec\Delta$, y además se puede demostrar que $(a_ia_{i+1})\land\Delta=a_i$ \cite{Thurston} para todo $i=1,\dots, r-1$. Esta es la anteriormente mencionada forma normal a la izquierda de la trenza. Los prefijos positivos de $\Delta$ son llamados \emph{elementos simples} o \emph{trenzas de permutación}. El nombre no es casual, ya que como prueba Thurston \cite{Thurston}, estas trenzas son justamente las mismas trenzas de permutación definidas en \ref{simples}. Por tanto, la forma normal a la izquierda de una trenza es una descomposición única como producto de una potencia de $\Delta$ y una sucesión de elementos simples propios. Thurston \cite{Thurston} mostró que esta forma normal puede ser calculada en tiempo $O(l^2n\log(n))$ para una palabra de $l$ letras en $B_n$. 

%prefijo propio de Delta porque Delta no era prefijo de A y se está haciendo el meet

En \cite{Thurston} se puede encontrar además una forma más práctica de llevar a cabo el algoritmo de encontrar la forma normal a la izquierda, la cual será la que utilicemos en el ejemplo \ref{ejnormal}. Antes de explicarla vamos a introducir algo de nomenclatura. Dadas dos trenzas simples positivas $A$ y $B$, decimos que un prefijo no trivial $b\preccurlyeq B$ \emph{se puede pasar} de $B$ a $A$ si $Ab$ es simple, y en tal caso \emph{pasar} $b$ de $B$ a $A$ consiste en las transformaciones $A\to Ab$ y $B\to b^{-1}B$. Con esto presente, el algoritmo consiste en lo siguiente:
\begin{enumerate}
\item Una vez tenemos una palabra $w\in B_n$ en forma normal de Garside $w=\Delta^p A$, si $A=1$, entonces no hay nada que hacer. En caso contrario, dividimos $A$ en bloques formados por elementos simples, digamos, $$A=a_{1,0}a_{2,0}\dots a_{m,0}.$$
\item En el paso $t\geq 0$ tenemos $A$ expresada en bloques de elementos simples como
$$A=a_{1,t}a_{2,t}\dots a_{m,t}.$$
En este paso buscamos el primer par $a_{i,t}a_{i+1,t}$ de modo que se pueda pasar algún prefijo de $a_{i+1,t}$ a $a_{i,t}$ y lo pasamos. Esto nos dará la descomposición 
$$A=a_{1,t+1}a_{2,t+1}\dots a_{m,t+1}.$$
\item Volvemos paso 2 y reiteramos hasta que no quede ningún par que verifique la condición. 
\end{enumerate} 

Este proceso naturalmente termina porque el vector formado por las longitudes de los bloques aumenta en cada paso su orden lexicográfico, el cual está acotado por $(m,0,\dots, 0)$ donde $m$ es la longitud de $A$. La forma normal a la izquierda se obtendrá eliminando los bloques triviales (que necesariamente estarán al final). %porque siempre se puede pasar una palabra no vacía a un bloque vacío

Alternativamente, podríamos empezar con una descomposición $w=\Delta^qA$ con $A\in B_n^+$, pero sin asegurarnos de que $\Delta \not\preccurlyeq A$, pues $\Delta$ aparecería al acumular elementos simples en caso de ser prefijo de $A$, y podríamos enviarlo al bloque de $\Delta^q$. En cualquier caso, este proceso acabará con la forma normal a la izquierda, pues no poder pasar ninguna letra del bloque $a_{i+1}$ al bloque $a_i$ es equivalente a que $a_i=(a_ia_{i+1})\land \Delta$. 

%porque poder pasar algo de i+1 a i es que se pueda coger un trozo de palabra más grande de (i i+1), es decir, sería un prefijo más grande de dicha palabra que i. Además debe ser prefijo de delta porque se está haciendo el meet, y estos son justamente los elementos simples. Como es maximal por ser el meet po ya está



\begin{ej}\label{ejnormal}
En $B_4$ sean $\alpha_1=\sigma_1\sigma_2^{-1}\sigma_3$ y $\alpha_2=\sigma_3\sigma_1\sigma_1\sigma_2\sigma_1$, las cuales queremos comprobar si representan el mismo elemento. Lo primero que debemos hacer es eliminar el exponente negativo de $\alpha_1$. Para ello, tenemos que expresar $\Delta=\Delta_4=\sigma_1(\sigma_2\sigma_1)(\sigma_3\sigma_2\sigma_1)$ de forma que tenga a $\sigma_2$ como sufijo. Esto es sencillo pues basta usar la técnica de la demostración del primer apartado de la Proposición \ref{sufijos} para escribir
\[
\Delta=\sigma_1(\sigma_2)(\sigma_3\sigma_2\sigma_1)\sigma_2.
\]
Así pues, $\sigma_2^{-1}=\Delta^{-1}\sigma_1\sigma_2\sigma_3\sigma_2\sigma_1$, de modo que $\alpha_1=\sigma_1\Delta^{-1}\sigma_1\sigma_2\sigma_3\sigma_2\sigma_1\sigma_3$. Usando la Proposición \ref{conjuga}, pasamos $\Delta^{-1}$ a la izquierda:
\[
\alpha_1=\Delta^{-1}\sigma_3\sigma_1\sigma_2\sigma_3\sigma_2\sigma_1\sigma_3
\]

Ahora vamos a hacer la separación de bloques en las palabras positivas. Empezamos con $\alpha_2$. Vamos a dividirla en los bloques $b_{1,0}=\sigma_3\sigma_1$ y $b_{2,0}=\sigma_1\sigma_2\sigma_1$, que son claramente trenzas simples. En general se puede comenzar por bloques de una sola letra. Así, obtenemos
\[
\alpha_2=b_{1,0}b_{2,0}=(\sigma_3\sigma_1)(\sigma_1\sigma_2\sigma_1).
\]
Aparentemente no podemos pasar ninguna letra de $b_{2,0}$ a $b_{1,0}$, pues aparecería $\sigma_1$ dos veces seguidas. Sin embargo, recordemos que las relaciones de \ref{presentacion} nos dan $\sigma_1\sigma_2\sigma_1=\sigma_2\sigma_1\sigma_2$. Por lo tanto, reescribimos $\alpha_2$ y continuamos
\begin{align*}
\alpha_2&=b_{1,1}b_{2,1}=(\sigma_3\sigma_1\sigma_2\sigma_1)(\sigma_2).
\end{align*}
Ahora tenemos la situación inversa: aparentemente podríamos añadir $\sigma_2$ al primer bloque, pero utilizando la misma relación de la presentación del grupo de trenzas que antes, nos aparecería $\sigma_2$ dos veces consecutivas, por lo que hemos finalizado el proceso y $\alpha_2=\Delta^0b_1b_2$ con $b_1=b_{1,1}$ y $b_2=b_{2,1}$. Obsérvese que el bloque que hemos pasado a la izquierda ($\sigma_2\sigma_1$) se corresponde con $b_{2,0}\land (b_{1,0}^{-1}\Delta)$ y el bloque resultante $(\sigma_3\sigma_1\sigma_2\sigma_1)$ se corresponde con $\alpha_2\land\Delta$ en el algoritmo original de El-rifai y Morton. Además es claro que ninguno de los factores es una potencia de $\Delta$.
% En el caso del primer factor podemos verlo mejor gráficamente, el número de cruces es distinto. Aunque mejor el número de factores, porque el producto de deltas no puede decrecer en letras

%\begin{figure}[h!]
%\centering
%\begin{tikzpicture}
%\braid[rotate=90,width=.65cm,height=0.7cm,line width=1.5pt] s_3^{-1} s_1^{-1} s_2^{-1} s_1^{-1};
%\end{tikzpicture}
%\caption{$\sigma_3\sigma_1\sigma_2\sigma_1$.}
%\end{figure}

%el bloque que hemos pasado a la izquierda se corresponde con eso porque estamos añadiendo elementos de Delta que no están en b_{1,0}

Vamos ahora con la parte positiva de $\alpha_1$, que la dividimos en bloques $a_{1,0}=\sigma_3$, $a_{2,0}=\sigma_1\sigma_2\sigma_3\sigma_2\sigma_1$ y $a_{3,0}=\sigma_3$. Tenemos 
\begin{align*}
\alpha_1&=(\sigma_3)(\sigma_1\sigma_2\sigma_3\sigma_2\sigma_1)(\sigma_3)=a_{1,0}a_{2,0}a_{3,0} \\
&=(\sigma_3\sigma_1\sigma_2\sigma_3)(\sigma_2\sigma_1)(\sigma_3)=a_{1,1}a_{2,1}a_{3,1}\\
&=(\sigma_3\sigma_1\sigma_2\sigma_3)(\sigma_2\sigma_1\sigma_3)()= a_{1,2}a_{2,2}a_{3,2}
\end{align*}
Vemos que ya no podemos pasar ninguna letra más a la izquierda y que ningún factor es una potencia de $\Delta$, por lo que hemos terminado. Como $a_{3,2}$ es trivial podemos eliminarlo, con lo que $\alpha_1=\Delta^{-1}a_1a_2$ donde $a_1=a_{1,2}$ y $a_2=a_{2,2}$. Al comparar las descomposiciones finales de $\alpha_1$ y $\alpha_2$ comprobamos que nos tienen la misma forma normal, luego representan elementos distintos.
\end{ej}
%lo que hace Garside es que como A es positiva ya sabes que tienes una cantidad finita de representaciones, entonces te vas a la que tenga Delta y reduces hasta que ninguna tenga Delta y entonces ya comparas. Para la left-greedy no hay que pasar por esto, directamente se calculan los meet y te acaba quedando el Delta a la izquierda y los factores simples 

Antes de terminar este capítulo, como comentábamos al final de la sección anterior, la forma normal de Garside permite probar la existencia y unicidad de mínimo común múltiplo y máximo común divisor en $B_n$ con el orden parcial inducido por el orden parcial definido en $B_n^+$.  

\begin{prop}
Dadas $a,b\in B_n$, existen $c=a\land b\in B_n$ y $d=a\lor b\in B_n$, es decir, existen el máximo común divisor y el mínimo común múltiplo en $B_n$. 
\end{prop}
\begin{dem}
Dadas $a,b\in B_n$, sean sus formas normales $a=\Delta^{p_1}A$ y $b=\Delta^{p_2}B$, donde $p_1,p_2\in\Z$ y $A,B\in B_n^+$. Tomando $q=\max\{|p_1|,|p_2|\}$ tenemos que $\Delta^qa,\Delta^qb\in B_n^+$. Por tanto, sabemos que existe un máximo común divisor $c=(\Delta^qa\land\Delta^qb)\in B_n^+$. Entonces, por la invarianza por multiplicación a izquierda del orden de prefijos
\begin{align*}
c\preccurlyeq \Delta^qa &\Leftrightarrow\Delta^{-q}c\preccurlyeq a,\\
c\preccurlyeq \Delta^qb &\Leftrightarrow\Delta^{-q}c\preccurlyeq b.
\end{align*}
Esto significa que $\Delta^{-q}c$ es un prefijo común de $a$ y de $b$. Sea $d\in B_n$ con $d\preccurlyeq a$ y $d\preccurlyeq b$. Existe entonces $N\geq q$ de modo que $\Delta^N d, \Delta^Na, \Delta^N b\in B_n^+$ y, además, $\Delta^N d \preccurlyeq\Delta^Na$ y $\Delta^N d \preccurlyeq\Delta^Nb$. Por tanto, 
$$\Delta^Nd\preccurlyeq \Delta^N a\land\Delta^N b=\Delta^{N-q}(\Delta^qa\land \Delta^qb)=\Delta^{N-q}c.$$
Por consiguiente, $d\preccurlyeq \Delta^{-q}c$, con lo que $\Delta^{-q}c$ es de hecho el máximo común divisor de $a$ y $b$. Análogamente se prueba para el mínimo común múltiplo. \QED
\end{dem}

%La invarianza por multiplicación del orden inducido en B_n se tiene por la definición
\end{document}

