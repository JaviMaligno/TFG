\documentclass[TFG.tex]{subfiles}

\begin{document}



\chapter{Formas normales}

El objetivo de este capítulo será proporcionar una \emph{forma normal} para las trenzas, es decir, una forma ``estándar'' de escribirlas, de modo que para ver si dos palabras son iguales sea suficiente calcular sus formas normales y comprobar si son iguales. Para llegar hasta esa forma normal estudiaremos la \emph{estructura de Garside} del grupo de trenzas. Todos los resultados de estea capítulo que no se demuestren explícitamente pueden encontrarse probados en \cite{Garside}, salvo que se especifique lo contrario.

\section{Estructura de Garside}

%\url{http://chaoxuprime.com/posts/2011-07-05-garside-normal-form-and-summit-sets.html}


Obsérvese que la presentación \ref{presentacion} solo involucra potencias positivas de los generadores. Por tanto, se puede considerar el monoide $B_n^+$ determinado por esa misma presentación. Los elementos de $B_n^+$ son palabras en $\sigma_1,\dots,\sigma_{n-1}$ (pero no sus inversos), y dos palabras son equivalentes si y solo si una puede obtenerse de la otra reemplazando reiteradamente subpalabras de la forma $\sigma_i\sigma_j$ con $|i-j|>1$ (respectivamente, $\sigma_i\sigma_j\sigma_i$ con $|i-j|=1$) por $\sigma_j\sigma_i$ (respectivamente, $\sigma_j\sigma_i\sigma_j$).

\begin{defi}
El monoide $B_n^+$ se denomina \emph{monoide de las trenzas positivas} y sus palabras son llamadas \emph{trenzas positivas}.
\end{defi}
En el monoide $B_n^+$ hay un orden parcial natural. 
\begin{defi}
Definimos en $B_n^+$ el orden parcial $\preccurlyeq$ tal que dadas $a,b\in B_n^+$, $a\preccurlyeq b$ si $ac=b$ para alguna $c\in B_n^+$. Decimos en ese caso que $a$ es un \emph{prefijo} de $b$. Escribimos $a\prec b$ si ni $c$ no es trivial. Si además $a\neq 1$, decimos que $a$ es un \emph{prefijo propio} de $b$. 
\end{defi}
Nótese que $\preccurlyeq$ es invariante por multiplicación a la izquierda, esto es, $a\preccurlyeq b$ implica $xa\preccurlyeq xb$ para todo $a,b,x\in B_n^+$. %vale el mismo c

Dado tal orden parcial, uno podría preguntarse si existe un único máximo común divisor o mínimo común múltiplo con respecto a $\preccurlyeq$. Esto es, dadas $a,b\in B_n^+$, ¿existe un único $d\in B_n^+$ tal que $d\preccurlyeq a$, $d\preccurlyeq b$ y $d'\preccurlyeq d$ para todo $d'$ prefijo común de $a$ y $b$? ¿Y existe un único $m\in B_n^+$ tal que $a\preccurlyeq m$, $b\preccurlyeq m$ y $m\preccurlyeq m'$ para todo $m'$ que tenga a $a$ y a $b$ como prefijos? En tales casos, escribimos $d=a\land b$ y $m=a\lor b$. Nótese que también tenemos $xd=xa\land xb$ y $xm=xa\lor xb$ para todo $x\in B_n^+$.

\begin{nota}
Análogamente podríamos definir el orden parcial dual  de \emph{sufijos}, $\succcurlyeq$, invariante por multiplicación a la derecha. Nótese que este orden no es equivalente al de prefijos, puesto que $b\succcurlyeq a$ no implica en general $a\preccurlyeq b$ ni recíprocamente. 
\end{nota}

El punto clave en el trabajo de Garside fue demostrar, mediante métodos elementales, que $\sigma_i$ y $\sigma_j$ tienen mínimo común múltiplo en $B_n^+$. En concreto:
\begin{prop} 
El mínimo común múltiplo de los generadores $\sigma_i$ y $\sigma_j$ viene dado por
$$\sigma_i\lor\sigma_j=\begin{cases}
\sigma_i\sigma_j & |i-j|>1,\\
\sigma_i\sigma_j\sigma_i & |i-j|=1.
\end{cases}$$
\end{prop}
%1.2 GARSIDE, la unicidad se ve fácil por la longitud

Él prueba al mismo tiempo que $B_n^+$ es cancelativo, es decir, $xay=xby$ implica $a=b$ para todo $a,b,x,y\in B_n^+$.

Como las relaciones de \ref{presentacion} son homogéneas, palabras equivalentes en $B_n^+$ tienen la misma longitud, por lo que la longitud de una palabra está bien definida en $B_n^+$. Aunque Garside no lo menciona explícitamente, inducción en esta longitud junto con la cancelatividad, permite probar a partir del resultado anterior que todo par de elementos tiene de $B_n^+$ tiene un único mínimo común múltiplo y un único máximo común divisor, tal como se prueba en \cite{Dehornoy}. 
%relaciones homogéneas= LA SUMA DE LOS EXPONENTES A UN LADO Y A OTRO DE LA ECUACIÓN ES LA MISMA
%lo de dehornoy lo tengo en el mismo papel donde me explica el 1.2 de Garside y lo de que los sigma_i son sufijos de Delta. Lo del mcm es lo de las flechitas y el mcd es los bloques

Garside después estudia el siguiente elemento especial.
\begin{defi}
La \emph{trenza fundamental} en $n$ hilos es la trenza
$$\Delta_n=\sigma_1(\sigma_2\sigma_1)\cdots(\sigma_{n-1}\dots\sigma_1).$$
Cuando $n$ se sobreentiende, utilizamos simplemente $\Delta$. 
\begin{figure}[h!]
\centering
\begin{tikzpicture}
\braid[rotate=90,width=.65cm,height=0.7cm,line width=1.5pt] s_1^{-1} s_2^{-1} s_1^{-1} s_3^{-1} s_2^{-1} s_1^{-1} s_4^{-1} s_3^{-1} s_2^{-1} s_1^{-1};
\end{tikzpicture}
\caption{$\Delta_5$.}
\end{figure}

\end{defi} 
A continuación, prueba el siguiente resultado.
\begin{prop}\label{conjuga}
Se verifican: 
\begin{enumerate}
\item $\Delta=\sigma_1\lor\dots\lor\sigma_{n-1}$.
\item $\sigma_i\Delta=\Delta\sigma_{n-i}$ para todo $i=1,\dots, n-1$.
\end{enumerate}
\end{prop} 
%ESTE RESULTADO TAMBIÉN. LA SEGUNDA ES EL LEMA 2, PERO LA PRIMERA NO SÉ. INTUYO QUE VER QUE ES MÚLTIPLO DE TODAS SE PUEDE HACER SIMPLEMENTE CON LAS RELACIONES

A partir de este resultado podemos deducir las siguientes propiedades sobre $\Delta$.

\begin{prop} Se cumplen:
\begin{enumerate}
\item $\sigma_1,\dots,\sigma_{n-1}$ son también sufijos de $\Delta$.
\item $\Delta^2$ conmuta con todo elemento de $B_n^+$.
\item Para todo $a\in B_n^+$ se tiene $a\preccurlyeq\Delta^m$ y $\Delta^m\succcurlyeq a$ para algún $m\geq 0$.
\end{enumerate}
%Esto implica que $\sigma_1,\dots,\sigma_{n-1}$ son también sufijos de $\Delta$, que $\Delta^2$ conmuta con todo elemento de $B_n^+$ y, por inducción en la longitud, que para todo $a\in B_n^+$ se tiene $a\preccurlyeq\Delta^m$ y $\Delta^m\succcurlyeq a$ para algún $m\geq 0$.  
\end{prop}
\begin{dem}\
\begin{enumerate}
\item En primer lugar, se tiene para $k>i\geq 1$ que $\sigma_i(\sigma_k\cdots\sigma_1)=(\sigma_k\cdots\sigma_1)\sigma_{i+1}$. En efecto, usando las relaciones del monoide de trenzas positivas
\begin{align*}
\sigma_i(\sigma_k\cdots\sigma_1)&=\sigma_i(\sigma_k\cdots\sigma_{i+2})(\sigma_{i+1}\sigma_i)(\sigma_{i-1}\cdots\sigma_1)\\
&=(\sigma_k\cdots\sigma_{i+2})(\sigma_i\sigma_{i+1}\sigma_i)(\sigma_{i-1}\cdots\sigma_1)\\
&=(\sigma_k\cdots\sigma_{i+2})(\sigma_{i+1}\sigma_i\sigma_{i+1})(\sigma_{i-1}\cdots\sigma_1)\\
&=(\sigma_k\cdots\sigma_{i+2})(\sigma_{i+1}\sigma_i)(\sigma_{i-1}\cdots\sigma_1)\sigma_{i+i}\\
&=(\sigma_k\cdots\sigma_1)\sigma_{i+1}.
\end{align*}
Así pues, para expresar $\sigma_i$ como sufijo de $\Delta$ hacemos los siguiente. Si $i=1$, entonces por la definición de $\Delta$ ya tenemos que es un sufijo. Si $1<i\leq n-1$, partimos de
\[
\Delta=\sigma_1(\sigma_2\sigma_1)\cdots(\sigma_{n-i}\dots\underline{\sigma_1})\cdots(\sigma_{n-1}\dots\sigma_1).
\]
Procedemos a desplazar a la derecha el $\sigma_1$ subrayado tal como hemos anteriormente. En cada paso irá aumentando el índice en una unidad. Por tanto, como hay $n-(n-i-1)=i-1$ bloques que se dejan atrás, obtenemos $\sigma_{1+i-1}=\sigma_i$, es decir,
\[
\Delta=\sigma_1(\sigma_2\sigma_1)\cdots(\sigma_{n-i}\dots\underline{\sigma_1})(\sigma_{n-i+1}\cdots\sigma_1)\cdots(\sigma_{n-1}\dots\sigma_1)=
\]
\[
\sigma_1(\sigma_2\sigma_1)\cdots(\sigma_{n-i}\cdots\sigma_2)(\sigma_{n-i+1}\cdots\sigma_1\underline{\sigma_2})\cdots(\sigma_{n-1}\dots\sigma_1)=
\]
\[
\sigma_1(\sigma_2\sigma_1)\cdots(\sigma_{n-1}\dots\sigma_1)\underline{\sigma_i}.
\]

\item Basta probar que $\Delta^2$ conmuta con $\sigma_i$ para todo $1\leq i\leq n-1$. 
\[
\sigma_i\Delta=\Delta\sigma_{n-i}\Leftrightarrow \sigma_i\Delta\Delta=\Delta(\sigma_{n-i}\Delta)\Leftrightarrow \sigma_i\Delta^2=\Delta^2\sigma_i.
\]
%Doble implicación por ser cancelativo

\item Probamos $a\preccurlyeq\Delta^m$ para algún $m\geq 0$ por inducción en la longitud. Evidentemente, $1\preccurlyeq\Delta$. Para una palabra de longitud 1 también es claro porque $\sigma_i\preccurlyeq\Delta$ para todo $1\leq i\leq n-1$ por definición de mínimo común múltiplo. Supongamos ahora, que para una palabra $a\in B_n^+$ de longitud $l-1$ se tiene el resultado. Entonces, cualquier palabra de longitud $l$ será de la forma $\sigma_j a$ para algún $1\leq j\leq n-1$. Así que, usando la invarianza por multiplicación a la izquierda y el caso $l=1$,
\[
a\preccurlyeq\Delta^m\Rightarrow \sigma_j a\preccurlyeq \sigma_j\Delta^m \preccurlyeq \Delta\Delta^m=\Delta^{m+1}.
\]
De forma análoga usando la invarianza por multiplicación a la derecha se prueba que $\Delta^m\succcurlyeq a$ para algún $m\geq 0$.
\end{enumerate}
\QED
\end{dem}


% el a<\Delta^m es cuestión de que como \sigma_i<\Delta, luego ya solo hay que ir multiplicando a izquierda por el sigma_j correspondiente -> \sigma_j\sigma_i<\sigma_j\Delta<\Delta^2, etc El otro igual usando la multiplicación a la derecha

Esto tiene importantes implicaciones. Como todo par de elementos de $B_n^+$ tiene un múltiplo común y $B_n^+$ es cancelativo, las condiciones de Öre (\ref{condiciones}) implican que $B_n^+$ se incrusta en su grupo de fracciones, que es precisamente $B_n$. Por lo tanto, $B_n^+$ no es solamente un monoide definido algebraicamente, sino que puede ser considerado como un subconjunto de $B_n$ formado por las trenzas que pueden ser escritas solo con potencias positivas de los generadores. 

Las propiedades anteriores implican que el orden parcial $\preccurlyeq$ (respectivamente, $\succcurlyeq$) puede ser extendido a $B_n$ de la siguiente manera: $a\preccurlyeq b$ (resp. $a\succcurlyeq b$) si $ac=b$ (resp. $b=ca$) para algún $c\in B_n^+$. Esto da un orden parcial que es invariante por multiplicación a la izquierda (resp. a la derecha), y el cual admite un único mínimo común múltiplo y un único máximo común divisor. Este hecho podrá ser probado una vez definida la \emph{forma normal de Garside} en la sección a continuación.
% ¿QUE LA EXISTENCIA Y UNICIDAD SE MANTENGA EN EL GRUPO ES INMEDIATO?

\section{Solución al problema de la palabra}
Garside dio una nueva solución al problema de la palabra en los grupos de trenzas de la siguiente manera. Recordemos que para todo $i=1,\dots, n-1$ se tiene que $\sigma_i\preccurlyeq\Delta$, esto es, $\Delta=\sigma_iX_i$ para algún $X_i\in B_n^+$. Dada una trenza escrita como una palabra en $\sigma_1,\dots,\sigma_{n-1}$ y sus inversos, se puede reemplazar cada aparición de $\sigma_i^{-1}$ por $X_i\Delta^{-1}$. Conjugar una trenza positiva por $\Delta$ sigue dando una trenza positiva por la proposición \ref{conjuga} apartado 2, así que podemos mover todas las apariciones de $\Delta^{-1}$ a la izquierda, de la siguiente forma: si encontramos $\sigma_i\Delta^{-1}$, entonces por \ref{conjuga} sabemos que $\Delta\sigma_{i}=\sigma_{n-i}\Delta\Leftrightarrow  \sigma_i\Delta^{-1}=\Delta^{-1}\sigma_{n-i}$, por lo que podemos sustituir $\sigma_i\Delta^{-1}$ por $\Delta^{-1}\sigma_{n-i}$. Esto muestra que toda trenza puede ser escrita como $\Delta^p A$ para algún $p\in\Z$ y algún $A\in B_n^+$. Además, si $\Delta\preccurlyeq A$, podemos reemplazar $\Delta^p$ por $\Delta^{p+1}$ y $A$ por $\Delta^{-1}A$. Esto reduce la longitud de $A$, así que solo puede hacerse una cantidad finita de veces. Por tanto, toda trenza puede escribirse, \emph{de manera única}, como $\Delta^pA$, donde $p\in\Z$, $A\in B_n^+$ y $\Delta\not\preccurlyeq A$. 

%\Delta c=A, \Delta^p A = \Delta^{p+1} \Delta^{-1}A=\Delta^{p+1}c
\begin{defi}
En base a lo comentado en el párrafo anterior, definimos la \emph{forma normal de Garside} de una palabra $w\in B_n$ como $w=\Delta^pA$, donde $p\in\Z$, $A\in B_n^+$ y $\Delta\not\preccurlyeq A$. 
\end{defi}

%la forma normal es única (A puede ser una equivalente obviamente), pues Delta es único y los pasos para llegar a la forma normal están unívocamente determinados

Esta forma normal permite resolver el problema de la palabra, ya que se pueden enumerar todas las palabras positivas que representan la trenza positiva $A$, reiterando las relaciones del grupo de trenzas de todas las formas posibles. Esta fue la solución dada por Garside. Sin embargo, no es muy satisfactoria, ya que da lugar a un algoritmo altamente ineficiente.

%al ser relaciones positivas y tal (la longitud es fija), solo hay una cantidad finita de posibilidades

El-rifai y Morton \cite{EM} lo mejoraron definiendo la \emph{forma normal izquierda} de una trenza. Basta tomar la descomposición $\Delta^pA$ y después definir %izquierda porque se van pasando elementos a la izquierda
\begin{align*}
&a_1=A\land\Delta\\
&a_i=(a_{i-1}^{-1}\cdots a_1^{-1}A)\land\Delta,\ i=1,\dots, n.
\end{align*}
De esta forma, toda trenza puede ser escrita de manera única como:
$$\Delta^p a_1\cdots a_r,$$
donde $a_i$ es un prefijo propio de $\Delta$, es decir, $1\prec a_i\prec\Delta$, y además $(a_ia_{i+1})\land\Delta=a_i$, para todo $i=1,\dots, r$. Esta es la anteriormente mencionada forma normal izquierda de la trenza. Los prefijos positivos de $\Delta$ son llamados \emph{elementos simples} o \emph{trenzas de permutación}. El nombre no es casual, ya que como prueba Thurston \cite{Thurston}, estas trenzas son justamente las mismas trenzas de permutación definidas en \ref{simples}. Por tanto, la forma normal izquierda de una trenza es una descomposición única como producto de una potencia de $\Delta$ y una sucesión de elementos simples propios. Thurston \cite{Thurston} mostró que esta forma normal puede ser calculada en tiempo $O(l^2n\log(n))$ para una palabra de $l$ letras en $B_n$. 

%prefijo propio de Delta porque Delta no era prefijo de A y se está haciendo el meet

EXPLICAR EL ALGORITMO POR BLOQUES CITANDO QUE ESTÁ EN THURSTON. RECORDAR COMENTAR QUE SI LA PALABRA SE ESCRIBE DE OTRA FORMA ALOMEJOR PODEMOS METER LETRAS Y DECIR QUE EN EL EJEMPLO SE VE UN CASO. SI ALGÚN BLOQUE SE QUEDA VACÍO SE ELIMINA

\begin{ej}
En $B_4$ sean $\alpha_1=\sigma_1\sigma_2^{-1}\sigma_3$ y $\alpha_2=\sigma_3\sigma_1\sigma_1\sigma_2\sigma_1$, las cuales queremos comprobar si representan el mismo elemento. Lo primero que debemos hacer es eliminar el exponente negativo de $\alpha_1$. Para ello, tenemos que expresar $\Delta=\Delta_4=\sigma_1(\sigma_2\sigma_1)(\sigma_3\sigma_2\sigma_1)$ de forma que tenga a $\sigma_2$ como prefijo. Esto es sencillo pues, basta usar las relaciones del monoide de trenzas positivas para escribir
\[
\Delta=\sigma_2(\sigma_1\sigma_2)(\sigma_3\sigma_2\sigma_1).
\]
Así pues, $\sigma_2^{-1}=(\sigma_1\sigma_2)(\sigma_3\sigma_2\sigma_1)\Delta^{-1}$, de modo que $\alpha_1=\sigma_1(\sigma_1\sigma_2)(\sigma_3\sigma_2\sigma_1)\Delta^{-1}\sigma_3$. Usando la proposición \ref{conjuga}, pasamos $\Delta^{-1}$ a la izquierda:
\begin{align*}
\alpha_1&=\sigma_1(\sigma_1\sigma_2)(\sigma_3\sigma_2\sigma_1)\Delta^{-1}\sigma_3=\sigma_1\sigma_1\sigma_2\sigma_3\sigma_2\Delta^{-1}\sigma_3\sigma_3=\sigma_1\sigma_1\sigma_2\sigma_3\Delta^{-1}\sigma_2\sigma_3\sigma_3\\
 & =\sigma_1\sigma_1\sigma_2\Delta^{-1}\sigma_1\sigma_2\sigma_3\sigma_3=\sigma_1\sigma_1\Delta^{-1}\sigma_2\sigma_1\sigma_2\sigma_3\sigma_3=\sigma_1\Delta^{-1}\sigma_3\sigma_2\sigma_1\sigma_2\sigma_3\sigma_3\\
 & =\Delta^{-1}\sigma_3\sigma_3\sigma_2\sigma_1\sigma_2\sigma_3\sigma_3.
\end{align*}

Ahora vamos a hacer la separación de bloques en las palabras positivas. Empezamos con $\alpha_2$. Vamos a dividirla en los bloques $a_1=\sigma_3\sigma_1$ y $a_2=\sigma_1\sigma_2\sigma_1$. Así, obtenemos
\[
\alpha_2=a_1a_2=(\sigma_3\sigma_1)(\sigma_1\sigma_2\sigma_1).
\]
Aparentemente no podemos pasar ninguna letra de $a_2$ a $a_1$, pues aparecería $\sigma_1$ dos veces seguidas. Sin embargo, recordemos que las relaciones de \ref{presentacion} nos dan $\sigma_1\sigma_2\sigma_1=\sigma_2\sigma_1\sigma_2$. Por lo tanto, reescribimos $\alpha_2$ y continuamos
\begin{align*}
\alpha_2&=(\sigma_3\sigma_1)(\sigma_2\sigma_1\sigma_2)=(\sigma_3\sigma_1\sigma_2\sigma_1)(\sigma_2).
\end{align*}
Ahora tenemos la situación inversa: aparentemente podríamos añadir $\sigma_2$ al primer bloque, pero utilizando la misma relación de la presentación del grupo de trenzas que antes, nos aparecería $\sigma_2$ dos veces consecutivas, por lo que hemos finalizado el proceso. Obsérvese que el bloque que hemos pasado a la izquierda se corresponde con $a_2\land (a_1^{-1}\Delta)$ y el bloque resultante se corresponde con $\alpha_2\land\Delta$ en el algoritmo original de El-rifai y Morton. Además es claro que ninguno de los factores es una potencia de $\Delta$.
% En el caso del primer factor podemos verlo mejor gráficamente, el número de cruces es distinto. Aunque mejor el número de factores, porque el producto de deltas no puede decrecer en letras
%\begin{figure}[h!]
%\centering
%\begin{tikzpicture}
%\braid[rotate=90,width=.65cm,height=0.7cm,line width=1.5pt] s_3^{-1} s_1^{-1} s_2^{-1} s_1^{-1};
%\end{tikzpicture}
%\caption{$\sigma_3\sigma_1\sigma_2\sigma_1$.}
%\end{figure}

Vamos ahora con la parte positiva de $\alpha_1$, que la dividimos en bloques $a_1=\sigma_3$, $a_2=\sigma_3\sigma_2\sigma_1\sigma_2\sigma_3$ y $a_3=\sigma_3$. Tenemos
\begin{align*}
\alpha_1&=a_1a_2a_3=(\sigma_3)(\sigma_3\sigma_2\sigma_1\sigma_2\sigma_3)(\sigma_3)\\
&=(\sigma_3)(\sigma_3\sigma_1\sigma_2\sigma_1\sigma_3)(\sigma_3)\\
&=(\sigma_3)(\sigma_1\sigma_3\sigma_2\sigma_1\sigma_3)(\sigma_3)\\
&=(\sigma_3\sigma_1)(\sigma_3\sigma_2\sigma_1\sigma_3)(\sigma_3)\\
&=(\sigma_1\sigma_3)(\sigma_3\sigma_2\sigma_3\sigma_1)(\sigma_3)\\
&=(\sigma_1\sigma_3)(\sigma_2\sigma_3\sigma_2\sigma_1)(\sigma_3)\\
&=(\sigma_1\sigma_3\sigma_2\sigma_3)(\sigma_2\sigma_1)(\sigma_3)\\
&=(\sigma_1\sigma_3\sigma_2\sigma_3)(\sigma_2\sigma_1\sigma_3)
\end{align*}
Vemos que ya no podemos pasar ninguna letra más a la izquierda y que ningún factor es una potencia de $\Delta$, por lo que hemos terminado y $\alpha_1=\Delta^{-1}(\sigma_1\sigma_3\sigma_2\sigma_3)(\sigma_2\sigma_1\sigma_3)$. Al comparar las descomposiciones finales de $\alpha_1$ y $\alpha_2$ comprobamos que nos tienen la misma forma normal, luego representan elementos distintos.
\end{ej}
%lo que hace Garside es que como A es positiva ya sabes que tienes una cantidad finita de representaciones, entonces te vas a la que tenga Delta y reduces hasta que ninguna tenga Delta y entonces ya comparas. Para la left-greedy no hay que pasar por esto, directamente se calculan los meet y te acaba quedando el Delta a la izquierda y los factores simples 

Antes de terminar este capítulo, como comentábamos al final de la sección anterior, la forma normal de Garside permite probar la existencia y unicidad de mínimo común múltiplo y máximo común divisor en $B_n$ con el orden parcial inducido por el orden parcial definido en $B_n^+$.  Dadas $a,b\in B_n$, sean sus formas normales $a=\Delta^{p_1}A$ y $b=\Delta^{p_2}B$, donde $p_1,p_2\in\Z$ y $A,B\in B_n^+$. Si $p_1,p_2\geq 0$ entonces $a,b\in B_n^+$ y no hay nada que probar. Si $p_1< 0$, entonces $\Delta^{-p_1}a, \Delta^{-p_1}b\in B_n^+$ y, similarmente, si $p_2<0$, entonces $\Delta^{-p_2}a, \Delta^{-p_2}b\in B_n^+$. Si  $p_1,p_2<0$, entonces tomamos $q=\max\{-p_1,-p_2\}$, de modo que $\Delta^{q}a, \Delta^{q}b\in B_n^+$. 

En cualquier caso, dadas $a,b\in B_n$, siempre existe $q\geq 0$ tal que $\Delta^qa,\Delta^qb\in B_n^+$. Por tanto, sabemos que existe un máximo común divisor $c=(\Delta^qa\land\Delta^qb)\in B_n^+$. Entonces, por la invarianza por multiplicación a la izquierda del orden de prefijos
\begin{align*}
c\preccurlyeq \Delta^qa &\Leftrightarrow\Delta^{-q}c\preccurlyeq a,\\
c\preccurlyeq \Delta^qb &\Leftrightarrow\Delta^{-q}c\preccurlyeq b.
\end{align*}
Esto implica que $\Delta^q(a\land b)=(\Delta^q a\land \Delta^qb)=c$, por lo que $a\land b$ existe y es único. Análogamente se prueba para el mínimo común múltiplo.

\end{document}

